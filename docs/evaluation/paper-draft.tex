\documentclass[conference]{IEEEtran}
\IEEEoverridecommandlockouts
\usepackage{cite}
\usepackage{amsmath,amssymb,amsfonts}
\usepackage{graphicx}
\usepackage{textcomp}
\usepackage{xcolor}
\usepackage{booktabs}
\usepackage{multirow}

\begin{document}

\title{Automatic Domain Randomization for Robust Sim-to-Real Transfer in Locomotion Tasks: Implementation and Parameter Relevance Analysis}

\author{\IEEEauthorblockN{Marc'Antonio Lopez}
\IEEEauthorblockA{\textit{Department of Control and Computer Engineering} \\
\textit{Polytechnic University of Turin}\\
Turin, Italy \\
s336362@studenti.polito.it}
\and
\IEEEauthorblockN{Luigi Marguglio}
\IEEEauthorblockA{\textit{Department of Control and Computer Engineering} \\
\textit{Polytechnic University of Turin}\\
Turin, Italy \\
s332575@studenti.polito.it}
}

\maketitle

\begin{abstract}
Transferring reinforcement learning policies from simulation to the real world remains a fundamental challenge due to the mismatch between simulated and physical dynamics, commonly known as the reality gap. In this work, we implement Automatic Domain Randomization (ADR) for the MuJoCo Hopper locomotion task and conduct a comprehensive two-part study. Part 1 evaluates ADR against baseline and Uniform Domain Randomization (UDR), finding that UDR achieves $+3.9\%$ positive transfer while ADR 10M achieves near-perfect stability ($-0.4\%$ gap). Part 2 presents a systematic ablation study analyzing the contribution of individual parameters (mass, damping, friction) to transfer performance. Our key finding is that \textbf{friction randomization alone} achieves a remarkable $+154.6\%$ transfer gap, outperforming all other configurations. Statistical analysis reveals that friction contributes $+68.7\%$ marginally (p=0.074), while mass shows a negative contribution ($-15.7\%$). Importantly, we discover strong antagonistic interaction effects: combining mass with friction reduces its positive effect by 94.5\%. These results suggest that selective parameter randomization based on relevance analysis outperforms uniform randomization of all parameters.
\end{abstract}

\begin{IEEEkeywords}
domain randomization, sim-to-real transfer, reinforcement learning, locomotion, ablation study, parameter relevance
\end{IEEEkeywords}

\section{Introduction}

Training robots in simulation offers significant practical advantages: unlimited data collection, massively parallel environments, and complete elimination of hardware damage risk. However, policies trained purely in simulation frequently fail when deployed on real hardware. This phenomenon, known as the reality gap, arises from the inevitable discrepancy between the idealized physics of simulation and the complex dynamics of the real world.

Domain Randomization addresses this challenge by varying simulation parameters during training. The underlying principle is straightforward: if an agent learns to perform well across a distribution of simulated environments, it should generalize to the real world, which can be viewed as just another sample from a sufficiently broad distribution. The robot effectively learns behaviors that are invariant to parameter changes, making it robust to the unknown real-world dynamics.

Traditional Uniform Domain Randomization (UDR) samples parameters from fixed ranges defined a priori. While this approach has achieved notable successes, it suffers from a fundamental limitation: selecting appropriate ranges requires careful manual tuning. Ranges that are too narrow may not encompass the real-world parameters, while ranges that are too wide can generate physically implausible scenarios, leading to a phenomenon called learned helplessness where the agent gives up learning anything useful.

Automatic Domain Randomization (ADR), introduced by OpenAI for dexterous manipulation, elegantly solves this problem by adapting randomization ranges based on agent performance. When performance exceeds a high threshold, the environment becomes ``too easy'' and ranges expand to increase difficulty. Conversely, when performance drops below a low threshold, ranges contract to make the task more manageable.

\textbf{Contributions.} This paper makes two main contributions:
\begin{enumerate}
    \item \textbf{Part 1 - ADR Evaluation:} Systematic comparison of ADR against baseline and UDR across multiple training durations.
    \item \textbf{Part 2 - Parameter Relevance Analysis:} Ablation study determining which parameters (mass, damping, friction) contribute most to transfer performance, including interaction effect analysis.
\end{enumerate}

\section{Related Work}

The sim-to-real transfer problem has been extensively studied in the robotics and reinforcement learning communities. Domain randomization was popularized by Tobin et al.~\cite{tobin2017} for visual tasks, demonstrating that policies trained on randomized synthetic images could transfer successfully to real cameras. The approach was subsequently extended to dynamics randomization by Peng et al.~\cite{peng2018}.

For locomotion specifically, Tan et al.~\cite{tan2018} achieved successful sim-to-real transfer for quadruped robots by randomizing friction coefficients and introducing latency randomization during training. OpenAI's work on solving Rubik's cube with a robot hand~\cite{openai2019} brought ADR to prominence by demonstrating that automatic difficulty adjustment could achieve unprecedented dexterity through an emergent curriculum.

Recent work by Gang et al.~\cite{gang2025} investigated the impact of static friction on sim-to-real transfer, finding that friction parameters are often the most critical for locomotion tasks. Our ablation study provides empirical support for this finding.

\section{Method}

\subsection{Environment and Task}

Our experiments use the MuJoCo Hopper environment, a standard benchmark for locomotion control. The agent observes an 11-dimensional state vector and outputs 3-dimensional continuous torques. We define a source environment with misspecified dynamics (1kg torso mass offset) and a target environment with correct dynamics.

\subsection{Automatic Domain Randomization}

Our ADR implementation maintains state $\mathcal{S} = \{\delta_m, \delta_d, \delta_f\}$ representing randomization ranges for mass, damping, and friction. Each $\delta \in [0, 1]$ specifies the fractional variation. The update rule: if mean reward $\bar{R} \geq R_{high}=1200$, ranges increase by $\epsilon = 0.05$; if $\bar{R} < R_{low}=600$, ranges decrease; otherwise unchanged.

\subsection{Ablation Study Design}

To analyze parameter relevance, we trained 10 configurations testing all combinations of mass (M), damping (D), and friction (F) randomization:

\begin{table}[htbp]
\caption{Ablation Study Configurations}
\begin{center}
\begin{tabular}{lccc}
\toprule
\textbf{Config} & \textbf{Mass} & \textbf{Damping} & \textbf{Friction} \\
\midrule
baseline & $\times$ & $\times$ & $\times$ \\
adr\_none & $\times$ & $\times$ & $\times$ \\
adr\_mass & $\checkmark$ & $\times$ & $\times$ \\
adr\_damp & $\times$ & $\checkmark$ & $\times$ \\
adr\_fric & $\times$ & $\times$ & $\checkmark$ \\
adr\_mass\_damp & $\checkmark$ & $\checkmark$ & $\times$ \\
adr\_mass\_fric & $\checkmark$ & $\times$ & $\checkmark$ \\
adr\_damp\_fric & $\times$ & $\checkmark$ & $\checkmark$ \\
adr\_all & $\checkmark$ & $\checkmark$ & $\checkmark$ \\
udr & $\checkmark$ & $\checkmark$ & $\checkmark$ \\
\bottomrule
\end{tabular}
\label{tab:configs}
\end{center}
\end{table}

All configurations used 2.5M timesteps and seed 42 for reproducibility.

\textbf{Note:} Part 1 and Part 2 are \textit{independent} experiments. Part 1 compares ADR at different durations (2.5M, 5M, 10M) against separately trained baseline/UDR models. Part 2 focuses on parameter ablation with a unified training framework, including its own baseline and UDR configurations for fair comparison within the ablation context.

\section{Part 1: ADR vs Baseline vs UDR}

\subsection{Training Dynamics}

Figure~\ref{fig:training} shows the ADR range expansion over training. Both the 2.5M and 5M runs achieved a final range of $\pm60\%$, while the 10M run plateaued at $\pm40\%$. This suggests that longer training does not necessarily lead to higher ADR ranges.

\begin{figure}[htbp]
\centering
\includegraphics[width=\columnwidth]{figures/training_curves.png}
\caption{ADR range expansion over training for different training durations.}
\label{fig:training}
\end{figure}

\subsection{Comparative Results}

Table~\ref{tab:part1} presents the main evaluation results from Part 1.

\begin{table}[htbp]
\caption{Part 1: Comparative Results (50 evaluation episodes, seed=42)}
\begin{center}
\begin{tabular}{lccc}
\toprule
\textbf{Method} & \textbf{Source} & \textbf{Target} & \textbf{Gap} \\
\midrule
Baseline & $1778 \pm 65$ & $1169 \pm 95$ & $-34.2\%$ \\
UDR & $1660 \pm 10$ & $\mathbf{1725 \pm 34}$ & $\mathbf{+3.9\%}$ \\
ADR 2.5M & $1567 \pm 7$ & $1533 \pm 133$ & $-2.1\%$ \\
ADR 5M & $1013 \pm 224$ & $781 \pm 139$ & $-22.9\%$ \\
ADR 10M & $1462 \pm 39$ & $1457 \pm 145$ & $-0.4\%$ \\
\bottomrule
\end{tabular}
\label{tab:part1}
\end{center}
\end{table}

Key findings:
\begin{itemize}
    \item \textbf{Baseline} exhibits severe reality gap ($-34.2\%$)
    \item \textbf{UDR} achieves best transfer ($+3.9\%$) with low variance
    \item \textbf{ADR 10M} achieves near-perfect stability ($-0.4\%$)
    \item \textbf{ADR 5M} shows poor performance despite high range
\end{itemize}

\section{Part 2: Ablation Study and Parameter Relevance}

\subsection{Transfer Performance by Configuration}

Table~\ref{tab:ablation} presents results from all 10 ablation configurations.

\begin{table}[htbp]
\caption{Part 2: Ablation Study Results (2.5M timesteps, seed=42)}
\begin{center}
\begin{tabular}{lccr}
\toprule
\textbf{Config} & \textbf{Source} & \textbf{Target} & \textbf{Gap} \\
\midrule
adr\_fric & $642 \pm 98$ & $\mathbf{1634 \pm 2}$ & $\mathbf{+154.6\%}$ \\
adr\_all & $1088 \pm 115$ & $1241 \pm 258$ & $+14.1\%$ \\
adr\_damp & $1631 \pm 301$ & $1761 \pm 31$ & $+8.0\%$ \\
adr\_damp\_fric & $1530 \pm 8$ & $1638 \pm 36$ & $+7.1\%$ \\
udr & $1724 \pm 10$ & $1711 \pm 104$ & $-0.8\%$ \\
adr\_mass\_fric & $1631 \pm 3$ & $1558 \pm 85$ & $-4.5\%$ \\
adr\_mass & $1180 \pm 181$ & $973 \pm 93$ & $-17.5\%$ \\
adr\_mass\_damp & $949 \pm 52$ & $648 \pm 29$ & $-31.7\%$ \\
adr\_none & $875 \pm 208$ & $303 \pm 222$ & $-65.4\%$ \\
baseline & $933 \pm 213$ & $314 \pm 224$ & $-66.4\%$ \\
\bottomrule
\end{tabular}
\label{tab:ablation}
\end{center}
\end{table}

The most striking result is that \texttt{adr\_fric} (friction-only randomization) achieves $+154.6\%$ transfer gap, vastly outperforming all other configurations including the full randomization approaches.

\begin{figure}[htbp]
\centering
\includegraphics[width=\columnwidth]{figures/ablation_transfer_gap.png}
\caption{Transfer gap by configuration. Positive values (green) indicate successful transfer; negative values (red) indicate reality gap.}
\label{fig:ablation}
\end{figure}

\subsection{Statistical Analysis of Parameter Contributions}

We computed the marginal contribution of each parameter using a factorial analysis approach. For each parameter $P$, we measured:
\begin{equation}
\text{Contribution}(P) = \bar{G}_{with\_P} - \bar{G}_{without\_P}
\end{equation}
where $\bar{G}$ is the mean transfer gap across relevant configurations.

\begin{table}[htbp]
\caption{Parameter Marginal Contributions}
\begin{center}
\begin{tabular}{lcrcc}
\toprule
\textbf{Rank} & \textbf{Param} & \textbf{Contrib.} & \textbf{p-value} & \textbf{Sig.} \\
\midrule
1 & FRICTION & $+68.70\%$ & 0.0744 & * \\
2 & MASS & $-15.68\%$ & 0.7120 & -- \\
3 & DAMPING & $-0.86\%$ & 0.9840 & -- \\
\bottomrule
\end{tabular}
\label{tab:contributions}
\end{center}
\end{table}

\textbf{Key Finding:} Friction is the only parameter with a positive marginal contribution, and it is marginally significant at $p < 0.10$. Mass shows a \textit{negative} contribution, meaning adding mass randomization tends to hurt transfer performance.

\subsection{Interaction Effects}

We analyzed whether parameter combinations produce synergistic or antagonistic effects beyond their individual contributions.

\begin{table}[htbp]
\caption{Parameter Interaction Effects}
\begin{center}
\begin{tabular}{lccc}
\toprule
\textbf{Interaction} & \textbf{Expected} & \textbf{Actual} & \textbf{Effect} \\
\midrule
M $\times$ D & $-11.1\%$ & $-6.1\%$ & $+5.0\%$ (syn.) \\
M $\times$ F & $+97.5\%$ & $+2.9\%$ & $\mathbf{-94.5\%}$ (ant.) \\
D $\times$ F & $+113.0\%$ & $+6.8\%$ & $\mathbf{-106.2\%}$ (ant.) \\
\bottomrule
\end{tabular}
\label{tab:interactions}
\end{center}
\end{table}

\textbf{Critical Discovery:} Strong antagonistic interactions exist. Adding mass to friction reduces friction's positive effect by 94.5\%. Similarly, adding damping to friction reduces its effect by 106.2\%. This explains why \texttt{adr\_fric} (friction only) dramatically outperforms \texttt{adr\_all} (all parameters).

\begin{figure}[htbp]
\centering
\includegraphics[width=\columnwidth]{figures/parameter_analysis.png}
\caption{Left: Parameter configuration matrix. Right: Transfer gap vs number of randomized parameters. Note the lack of monotonic relationship.}
\label{fig:param_analysis}
\end{figure}

\section{Discussion}

Our two-part study reveals several important insights:

\textbf{1. Friction Dominates Transfer Performance.} The exceptional performance of friction-only ADR ($+154.6\%$) can be attributed to:
\begin{itemize}
    \item Ground contact is critical for locomotion
    \item Friction is often the most misspecified parameter in simulation
    \item Single-parameter ADR allows focused robustness learning
\end{itemize}

\textbf{2. Mass Randomization Can Hurt.} Counter-intuitively, mass randomization shows negative contribution ($-15.7\%$). This may be because:
\begin{itemize}
    \item The source environment already has a 1kg torso mass offset
    \item Additional mass randomization interferes with this built-in mismatch
    \item Over-robustification leads to conservative behaviors
\end{itemize}

\textbf{3. More Parameters $\neq$ Better Transfer.} Strong antagonistic interactions mean that randomizing all parameters can be worse than selective randomization. This challenges the common assumption that broader randomization is always beneficial.

\textbf{4. ADR Range $\neq$ Transfer Quality.} ADR 5M achieved $\pm 60\%$ range but $-22.9\%$ gap, while ADR 10M achieved only $\pm 40\%$ range but $-0.4\%$ gap. The ablation study explains this: what matters is \textit{which} parameters are randomized, not just how much.

\subsection{Practical Recommendations}

Based on our empirical evidence:

\begin{table}[htbp]
\caption{Recommended Configurations by Use Case}
\begin{center}
\begin{tabular}{ll}
\toprule
\textbf{Goal} & \textbf{Recommendation} \\
\midrule
Maximum transfer & \texttt{adr\_fric} (friction only) \\
Balanced robustness & \texttt{adr\_damp\_fric} \\
Conservative approach & \texttt{udr} (fixed $\pm 30\%$) \\
\textbf{Avoid} & \texttt{adr\_mass}, \texttt{adr\_mass\_damp} \\
\bottomrule
\end{tabular}
\label{tab:recommendations}
\end{center}
\end{table}

\section{Conclusion}

We presented a comprehensive study of domain randomization for locomotion, encompassing both method comparison (Part 1) and parameter relevance analysis (Part 2). Our key findings:

\begin{enumerate}
    \item \textbf{Friction is king:} Friction-only ADR achieves $+154.6\%$ transfer gap, outperforming all other configurations.
    \item \textbf{Mass hurts:} Mass randomization contributes $-15.7\%$ to transfer, likely due to interference with built-in dynamics mismatch.
    \item \textbf{Interactions matter:} Strong antagonistic effects reduce combined parameter benefits by up to 106\%.
    \item \textbf{Selective randomization wins:} Data-driven parameter selection outperforms uniform randomization of all parameters.
\end{enumerate}

These results have practical implications: practitioners should conduct ablation studies to identify relevant parameters rather than assuming that more randomization is better.

\section*{Acknowledgment}

This work was conducted as part of the Robot Learning course at Politecnico di Torino, under the supervision of the VANDAL laboratory.

\begin{thebibliography}{00}
\bibitem{openai2019} OpenAI et al., ``Solving Rubik's Cube with a Robot Hand,'' arXiv:1910.07113, 2019.
\bibitem{tobin2017} J. Tobin et al., ``Domain Randomization for Transferring Deep Neural Networks from Simulation to the Real World,'' IROS, 2017.
\bibitem{peng2018} X. B. Peng et al., ``Sim-to-Real Robot Learning from Pixels with Progressive Nets,'' CoRL, 2018.
\bibitem{tan2018} J. Tan et al., ``Sim-to-Real: Learning Agile Locomotion For Quadruped Robots,'' RSS, 2018.
\bibitem{schulman2017} J. Schulman et al., ``Proximal Policy Optimization Algorithms,'' arXiv:1707.06347, 2017.
\bibitem{gang2025} S. Gang et al., ``Impact of Static Friction on Sim2Real in Robotic Reinforcement Learning,'' 2025.
\end{thebibliography}

\end{document}
